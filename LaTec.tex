\documentclass[12pt, letterpaper, twoside]{article}
\usepackage{graphicx}% LaTeX package to import graphics
\graphicspath{{images/}} % configuring the graphicx package

\title{My first LaTeX document}
\author{Raksha Raut}

\date{july 2024}
\begin{document}
\parindent0pt% To remove the paragraph indentation
\maketitle
% \includegraphics{raksha}
\begin{figure}[h]
    \centering
    \includegraphics[width=0.75\textwidth]{raksha}
    \caption{A Memory.}
    \label{raksha}

 
As you can see in figure \ref{raksha},this is a snap from tech parva2.0.
%\pageref{raksha}.
\end{figure}

%font styling

\begin{itemize}
  \item I am from \emph{Dhanusha},studying computer science at pashchimanchal collage.
  \item I'm from 2078 batch.
\end{itemize}

\begin{enumerate}
  \item \textit{I am from \emph{Dhanusha},studying computer science at pashchimanchal collage.}
  \item I'm from 2078 batch.

\end{enumerate}
\maketitle
  
\tableofcontents

\section{Introduction}
   
This is the first section.
      
Lorem  ipsum  dolor  sit  amet,  consectetuer  adipiscing  
elit. 
\section*{Unnumbered Section}
\addcontentsline{toc}{section}{Unnumbered Section}

Lorem ipsum dolor sit amet, consectetuer adipiscing elit.  

\section{Second Section}
       
Lorem ipsum dolor sit amet, consectetuer adipiscing elit.  



  \textbf{I am from  \emph{Dhanusha},studying computer science in pashchimanchal collage.}
 %MATH TIME 
%Inline math mode

In physics, the mass-energy equivalence is stated 
by the equation $E=mc^2$, discovered in 1905 by Albert Einstein.

\begin{math}
E=mc^2
\end{math} is typeset in a paragraph using inline math mode---as is $E=mc^2$, and so too is \(E=mc^2\).

%Display math mode
The mass-energy equivalence is described by the famous equation
\[ E=mc^2 \] discovered in 1905 by Albert Einstein. 

In natural units ($c = 1$), the formula expresses the identity
\begin{equation}
E=m
\end{equation}
% \\ || \newline for newline 


\section{Section3}

This is the first section.

\section{4th Section}

It seems like there might be a misunderstanding or a typo in the command syntax. The error message suggests that the command grv origin git@github.com:RakshaRaut/LatexStyles.git is not recognized as a valid Git command.\\It seems like there might be a misunderstanding or a typo in the command syntax. The error message suggests that the command grv origin git@github.com:RakshaRaut/LatexStyles.git is not recognized as a valid Git command.

To add a remote repository named origin using Oh-My-Zsh Git plugin, you should use the shorthand form directly:

\subsection{Table as Subsection}
%\end{document}

%    to add horizontal rules, above and below rows, use the \hline command
%    to add vertical rules, between columns, use the vertical line parameter |
%creating Tables
\begin{center}
\begin{tabular}{|c|c|c|} 
 \hline
 himaya1 & himaya2 & himaya3 \\ 
 boho4 & boho5 & boho6 \\ 
 hogo7 & hogo8 & hogo9 \\ 
 \hline
\end{tabular}
\end{center}


\end{document}
